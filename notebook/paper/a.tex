\documentclass{ctexart}
\title{人工智能工程伦理教育问题研究}
\author{王承厚SA19011162}
\begin{document}
\maketitle
\section{摘要}
随着科技不断发展,人工智能技术深刻改变了我们的生产生活,人类生活正朝着由互联网、大数据构成的信息化社会迈进,日常生活、购物出行都有人工智能技术帮我们减少时间成本、提高效率。但在带给我们便利的同时,给整个社会造成了道德伦理风险。从“中国人喜欢用隐私换取便利”到“会员价格歧视”,整个人工智能领域正面临价值观缺失和监管不到位的问题。作为培养未来社会中坚力量的高校,做好工程伦理教育,培养大学生工程伦理意识,掌握工程规范,是重要的课题。

\section{abstract}
With the continuous development of science and technology, artificial intelligence technology has profoundly changed our production and life, and human life is moving towards an information society composed of the Internet and big data. Daily life, shopping and travel have artificial intelligence technology to help us reduce time costs and improve efficiency. However, while bringing convenience to us, it has caused moral and ethical risks to the whole society. From "Chinese people like to trade privacy for convenience" to "member price discrimination", the whole field of artificial intelligence is facing the problem of lack of values and inadequate supervision. As the backbone of the future society, it is an important task for colleges and universities to do a good job in engineering ethics education, cultivate college students' engineering ethics awareness and master engineering norms.

\section{人工智能专业伦理问题}
计算机算力的提升带来了人工智能的飞速发展,取代了部分的繁重重复的工作,解放和发展了生产力。但数据信息从一方面解放了社会,又从另一方面奴役了社会。美团为了缩减成本、压缩配送时间,不断用算法试探骑手的极限配送时间,造成了外卖骑手与顾客之间冲突等负外部效益。另一个例子是科技巨头超时工作机制和不断发生的猝死案例。科技带来生产力的进步时,并不保证带来生产关系的改善。

依托于概率统计的一些人工智能技术带来了新的问题。基于大数据训练等识别感知算法不是确定性算法,不能保证完全正确,在保证经济可行下,认为某些牺牲和错误是可接受的。如果不预设道德伦理机制,将可能引起灾难性后果。一个例子就是福特公司计算成本收益分析后,宁可赔偿可能出现的汽车伤亡事故,也要选择出售有安全隐患的汽车。算法另一面就是带来歧视,这种歧视是不自觉的,来源于数据并被算法放大,使歧视长存于整个算法中,并且可能会不断加强,形成一个正反馈循环。在现实生活中,信用评估、风险评估都或多或少存在类似问题。

人工智能伦理研究包括人工智能的道德哲学、道德算法、设计伦理、社会伦理四个维度。而人工智能工程伦理则关注于人工智能技术理论发展和使用准则。人工智能工程伦理主要有三个原则:社会公正、人道主义、和谐发展。“社会主义”追求社会的公平公正、不能造成社会的贫富差距过大;“人道主义”原则追求工程与人的和谐,要求工程技术不能带给人的额外痛苦;“和谐发展”原则追求工程与自然的和谐发展,表明追求工程技术的发展不能以牺牲自然为代价。

\section{工程伦理教育存在的问题}
高校工程伦理教育是高校培养体系中不可或缺的一环,也是科技人才发展过程中的不可或缺的部分。在工程领域中,没有工程伦理的约束,工程技术这把双刃剑可能带给社会不小的负面影响。

在工程伦理教育体系中,无论是工程伦理课程数量,还是工程伦理课的质量都不能满足培养专业人才的需求。一方面是工程伦理教育起步较晚,尚未形成完整的授课体系,专业教育者不足,课堂上学生重视不足,不能发挥课程的效果。

根据相关调查,只有部分高校开设了工程伦理课程。在部分高校开设的工程伦理课程中,没有形成完整有效的课程体系。课程以思想教育为主,忽略了工程伦理思想的辩证思考,课程流于表面,使得学生在未来的工程实践中,没有工程伦理底线,选择利益。

\section{人工智能工程伦理教育发展}
人工智能技术是未来世界的发展方向,发展人工智能工程伦理教育应坚持以人为本,避免人工智能带来的道德风险,增进人类福祉为目标。工程伦理教育者应摆正心态,积极主动迎接人工技能技术的发展;相关专业的学生在学习人工智能技术的同时,学习工程伦理技术,培养工程伦理意识。在工程伦理的顶层设计中,应完善工程伦理教育体系的发展。

人工智能工程伦理教育是新交叉课程,应推动教育学、工程学和伦理学的融合。在课程学习中,既要注重课程的理论知识学习,又要注重课堂实践,引导学生对工程伦理内容的主动思辨。

建立人工智能工程伦理教育体系需要培养一支专业人才队伍。人工智能伦理教育人才需要多学科知识融合,包括但不限于人工智能技术、工程学、伦理学等学科。这需要高校之间主动交流,通过工程项目、交流研讨会等途径进行科研与教育要求,促进人工智能工程伦理教育的发展

\section{结论}
人工智能技术的发展需要相关工程伦理技术的制约,才能尽量避免双刃剑的负面效应。重视工程伦理教育是现在高校教育发展不可或缺的一部分,也是世界上高校工程教育的发展趋势。重视实践、主动引导学生对工程伦理的思辨、培养学生的工程伦理技术、塑造底线思维。从顶层设计上来讲,应正视工程伦理教育的不足,主动补齐短边,在教育部门、高校社团、工程企业的支持和配合下,培养一支优秀的工程伦理教育的专业人才,形成完善的工程伦理教育体系。

\end{document}
